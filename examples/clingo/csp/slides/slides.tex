\documentclass[11pt]{beamer}
\usetheme{Warsaw}
\usepackage[utf8]{inputenc}
\usepackage{amsmath}
\usepackage{amsthm}
\usepackage{amsfonts}
\usepackage{amssymb}
\usepackage{stmaryrd}
\usepackage{listings}
\author[Cabalar et al.]{Pedro Cabalar and Jorge Fandinno and Torsten Schaub and Philipp Wanko}
\title[Implementing $\mathit{HT}_C$ using CASP]{Implementing (part of) $\mathit{HT}_C$ using Constraint Answer Set Programming: A prototype}
%\setbeamercovered{transparent} 
%\setbeamertemplate{navigation symbols}{} 
%\logo{} 
%\institute{} 
\date{} 
%\subject{} 
\begin{document}

%!TEX root = paper.tex

\newtheorem{proposition}{Proposition}
\newcommand{\den}[1]{\llbracket \, #1 \, \rrbracket}
% \newcommand{\denc}[3]{\llbracket \, #3 \, \rrbracket_{\langle#1,#2\rangle}}
% \newcommand{\inter}[2]{\ensuremath{\mathcal{I}_{#1,#2}}}
% \newcommand{\CC}{\ensuremath{\mathcal{C^+}}} % {\ensuremath{\mathcal{C_C}}} % {\ensuremath{\mathcal{CC}}}
% \newcommand{\TC}[1]{\ensuremath{\mathcal{T_{#1}}}}
% \newcommand{\cond}[2]{\ensuremath{|#1:#2|}}
\newcommand{\ctermm}[3]{\ensuremath{{#1|#2}{:\,}#3}}
\newcommand{\cterm}[3]{\ensuremath{(\ctermm{#1}{#2}{#3})}} % {\ensuremath{({#1|#2}\mathrel{:}#3})}
% \newcommand{\ct}[1]{\ensuremath{\mathit{ct}(#1)}}
% \newcommand{\cte}[2]{\ensuremath{\mathit{ct}_{\langle #1\rangle}(#2)}}
\newcommand{\close}[1]{\ensuremath{#1\raisebox{1pt}{$\uparrow$}}}

\newcommand{\htag}[2]{\ensuremath{\mathit{#1}\agg{#2}}}
\newcommand{\htagg}[4]{\ensuremath{\mathit{#1}\agg{#2}\mathrel{#3}#4}}
\newcommand{\htaggg}[4]{\ensuremath{\mathit{#1} {#2 }\mathrel{#3}#4}}
\newcommand{\vals}[2]{\ensuremath{\mathcal{V}_{#1,#2}}}
\newcommand{\eval}[2]{\ensuremath{\mathit{eval}}_{\langle #1,#2\rangle}}
% \newcommand{\evalgz}[2]{\ensuremath{\mathit{eval}^{\gz}_{\langle #1,#2\rangle}}}
% \newcommand{\evalf}[2]{\ensuremath{\mathit{eval}^{\f}_{\langle #1,#2\rangle}}}
% \newcommand{\evals}[3]{\ensuremath{\mathit{eval}^{#1}_{\langle #2,#3\rangle}}}
% \newcommand{\evalcl}[1]{\ensuremath{\mathit{eval}_{#1}}}

\newcommand{\evalgz}[2]{\ensuremath{{\gz}_{\langle #1,#2\rangle}}}
\newcommand{\evalf}[2]{\ensuremath{{\f}_{\langle #1,#2\rangle}}}
\newcommand{\evals}[3]{\ensuremath{{#1}_{\langle #2,#3\rangle}}}
\newcommand{\evalcl}[1]{\ensuremath{\mathit{eval}_{#1}}}
\newcommand{\evall}[1]{\ensuremath{\mathit{eval}}_{#1}}
\newcommand{\evallgz}[1]{\ensuremath{\mathit{eval}^{\mathtt{gz}}_{#1}}}

\newcommand{\modelscl}{\ensuremath{\models_{cl}}}
\newcommand{\eqdef}{%
  \mathrel{\vbox{\offinterlineskip\ialign{%
    \hfil##\hfil\cr%
    $\scriptscriptstyle\mathrm{def}$\cr%
    \noalign{\kern1pt}%
    $=$\cr%
    \noalign{\kern-0.1pt}%
}}}}

\newcommand{\sysfont}{\textit}
\newcommand{\clingo}{\sysfont{clingo}}
\newcommand{\clingcon}{\sysfont{clingcon}}
\newcommand{\clingoDL}{\clingo{\small\textnormal{[}\textsc{DL}\textnormal{]}}}
\newcommand{\dlprogram}{\textit{DL}-program}
\newcommand{\code}[1]{\texttt{#1}}
\newcommand{\ground}{variable-free}
\newcommand{\modelsgz}{\models_{\gz}}
\newcommand{\modelsf}{\models_{\f}}
\newcommand{\modelss}{\models_{\sem}}
\newcommand{\modelsp}[1]{\models_{#1}}
\newcommand{\Atoms}{\mathit{Atoms}}
\newcommand{\Head}{\mathit{H}}
\newcommand{\HeadA}{\mathit{\Head_A}}
\newcommand{\HeadB}{\mathit{\Head_B}}
\newcommand{\Headp}{\mathit{\Head^+}}
\newcommand{\Headn}{\mathit{\Head^-}}
\newcommand{\Body}{\mathit{B}}
\newcommand{\Bodyp}{\mathit{\Body^+}}
\newcommand{\Bodyn}{\mathit{\Body^-}}
\newcommand{\var}{\mathit{var}}
\newcommand{\Cond}{\mathit{Cond}}
\newcommand{\tuple}[1]{\langle #1 \rangle}
\newcommand{\restr}[2]{{#1|}_{\hspace{-1pt}#2}}
\newcommand{\df}[1]{\mathit{def}(#1)}
\newcommand{\HTC}{\ensuremath{\mathit{HT}_{\!C}}} % {\mathit{HT}_C}
\newcommand{\HT}{\ensuremath{\mathit{HT}}}
\newcommand{\X}{\ensuremath{\mathcal{X}}}
\newcommand{\D}{\ensuremath{\mathcal{D}}}
\newcommand{\Du}{\ensuremath{\mathcal{D}_{\mathbf{u}}}}
\newcommand{\T}{\ensuremath{\mathcal{T}}}
\newcommand{\C}{\ensuremath{\mathcal{C}}}
\newcommand{\ET}{\ensuremath{\mathcal{T}^e}}
\newcommand{\BT}{\ensuremath{\mathcal{T}^b}}
\newcommand{\BC}{\ensuremath{\mathcal{C}^b}}
\newcommand{\Cgr}{\ensuremath{\mathcal{C}^*}}
\newcommand{\F}{\ensuremath{\mathcal{F}}}
\newcommand{\I}{\ensuremath{\mathcal{I}}}
\newcommand{\V}{\ensuremath{\mathcal{V}}}
%\newcommand{\P}{\ensuremath{\mathcal{P}}}
\newcommand{\A}{\ensuremath{\mathcal{A}}}
\newcommand{\Piref}[1]{\Pi_{\ref{#1}}}
\newcommand{\vars}[1]{\ensuremath{\mathit{vars}(#1)}}
\newcommand{\varsp}[1]{\ensuremath{\mathit{vars}^{a}(#1)}}
\newcommand{\atoms}[1]{\ensuremath{\mathit{atoms}(#1)}}
\newcommand{\At}[1]{\ensuremath{\mathit{At}(#1)}}
\newcommand{\val}{\ensuremath{v}} % {V}
% \newcommand{\t}{\ensuremath{\mathbf{t}}} % {\boldsymbol{t}}
\newcommand{\gz}{\ensuremath{\mathit{vc}}} % {\boldsymbol{f}}
\newcommand{\f}{\ensuremath{\mathit{df}}} % {\boldsymbol{f}}
\newcommand{\ff}{\ensuremath{\mathit{F}}} % {\boldsymbol{f}}
\newcommand{\undefined}{\ensuremath{\mathbf{u}}} % {\boldsymbol{u}}
\newcommand{\sem}{\ensuremath{\kappa}} % {\boldsymbol{u}}
\newcommand{\ass}[3]{#1 := #2 \, .. \, #3}
\newcommand{\LC}{\ensuremath{\mathit{LC}}}
\newcommand{\DF}{\ensuremath{\boldsymbol{D\hspace{-1.2pt}F}}} % {\mathbf{DF}}
\newcommand{\agg}[1]{\ensuremath{\dot{\{}#1\dot{\}}}} % {\text{\"\{}#1\text{\"\}}}
\newcommand{\Z}{\ensuremath{\mathbb{Z}}}
\newcommand{\LX}{\ensuremath{\mathbb{X}}}
\newcommand{\HU}{\ensuremath{\mathbb{U}}}
\newcommand{\Gra}{\ensuremath{\mathit{G}^a}}
\newcommand{\Gr}{\ensuremath{\mathit{G}}}
\newcommand{\Def}{\delta}
\newcommand{\grsep}{\,\big|\hspace{-3pt}\big|\hspace{-3pt}\big|\,}
\newcommand{\isint}[1]{\ensuremath{\mathit{int}(#1)}}




\let\olditem\item
\let\oldenumerate\enumerate
\let\oldendenumerate\endenumerate
\let\olditemize\itemize
\let\oldenditemize\enditemize
\newcommand{\deitemize}{%
\def\itemize{\def\item{}}
\def\enditemize{\let\item\olditem}
\def\enumerate{\let\item\olditem\oldenumerate}
\def\endenumerate{\oldendenumerate\def\item{}}
}

\newcommand{\reitemize}{%
\def\itemize{\olditemize}
\def\enditemize{\oldenditemize}
}


%%%%%%%%%%%%%%%%%%%%%%%%%%%%%%%%%%%%%%%%%%%%%%%%%%%%%%%%%%%%%%%%%%%%%%%%%%%%%%%%%%%%%%%%%%%%%%%%%
%%%%%%%%%%%%%%%%%%%%%%%%%%%%%%%%%%%%%%%%%%%%%%%%%%%%%%%%%%%%%%%%%%%%%%%%%%%%%%%%%%%%%%%%%%%%%%%%%
%%% with bullets

\newenvironment{Itemize}{\begin{itemize}[leftmargin=0pt]}{\end{itemize}}

%%%%%%%%%%%%%%%%%%%%%%%%%%%%%%%%%%%%%%%%%%%%%%%%%%%%%%%%%%%%%%%%%%%%%%%%%%%%%%%%%%%%%%%%%%%%%%%%%
%%%%%%%%%%%%%%%%%%%%%%%%%%%%%%%%%%%%%%%%%%%%%%%%%%%%%%%%%%%%%%%%%%%%%%%%%%%%%%%%%%%%%%%%%%%%%%%%%
%%% without bullets

\renewenvironment{Itemize}{\let\item\relax}{\let\item\olditemize}
\renewenvironment{itemize}{\let\item\olditem\olditemize}{\oldenditemize\let\item\relax}
\renewenvironment{enumerate}{\let\item\olditem\oldenumerate}{\oldendenumerate\let\item\relax}

%%%%%%%%%%%%%%%%%%%%%%%%%%%%%%%%%%%%%%%%%%%%%%%%%%%%%%%%%%%%%%%%%%%%%%%%%%%%%%%%%%%%%%%%%%%%%%%%%
%%%%%%%%%%%%%%%%%%%%%%%%%%%%%%%%%%%%%%%%%%%%%%%%%%%%%%%%%%%%%%%%%%%%%%%%%%%%%%%%%%%%%%%%%%%%%%%%%



% Some aggregate names
\def\sumf{\mathit{sum}}
\newcommand\op{\mathit{op}}
\newcommand\EX{\mathcal{EX}}

\newcommand{\appendblank}{\ensuremath{\circ}}
\newcommand\theory[2]{\tuple{#1,#2}}
\newcommand{\aggs}{\mathit{agg}}
\newcommand{\levels}{\ell}
\newcommand{\level}[1]{\levels(#1)}
\newcommand{\gzsemantics}{\mbox{\gz-semantics}\xspace}
\newcommand{\fsemantics}{\mbox{\f-semantics}\xspace}
\newcommand{\sequilibrium}{\mbox{\sem-equilibrium}\xspace}
\newcommand{\gzequilibrium}{\mbox{\gz-equilibrium}\xspace}
\newcommand{\fequilibrium}{\mbox{\f-equilibrium}\xspace}
\newcommand{\sstable}{\mbox{\sem-stable}\xspace}
\newcommand{\gzstable}{\mbox{\gz-stable}\xspace}
\newcommand{\fstable}{\mbox{\f-stable}\xspace}
\def\kmapping{selection function}
\def\kmappings{{\kmapping}s}
%%% Local Variables:
%%% mode: latex
%%% TeX-master: "paper"
%%% End:

% The language definitions below require the following packages:
%
%     \usepackage{courier}
%     \usepackage{xcolor}
%     \usepackage{relsize}
%
%     % The language definitions below require the following packages:
%
%     \usepackage{courier}
%     \usepackage{xcolor}
%     \usepackage{relsize}
%
%     % The language definitions below require the following packages:
%
%     \usepackage{courier}
%     \usepackage{xcolor}
%     \usepackage{relsize}
%
%     \input{listings}
%     \renewcommand{\Underscore}{\textscale{1}{\textunderscore}}
%
% Without extra packages I had to provide an alternative underscore definition
% to get decent looking underscores.
%
% As an alternative to courier, it is also possible to use the `lmodern`
% package.
%
%     \usepackage[T1]{fontenc}
%     \usepackage{lmodern}
%     \usepackage{xcolor}
%     \usepackage{relsize}
%
%     \input{listings}
%
% When the python or clingo language is given, tex code can be embedded
% escaping it in `#( ... #)`, which is treated like a comment in Python.
% Furthermore, for logic programs the character sequence is `%%` is not output.
% This can be used to add labels at the end of lines. For example, in a logic
% program one can write
%
%     H :- B.%%#( \label{lst:rule} #)
%
% to refer to the rule from the document without having to worry about changing
% line numbers when refactoring. The annotated program will still be runnable
% with clingo.

\providecommand{\Underscore}{\textunderscore}

\lstdefinelanguage{clingo}{%
  basicstyle=\ttfamily,%
  keywordstyle=[1]\bfseries,%
  keywordstyle=[2]\bfseries,%
  keywordstyle=[3]\bfseries,%
  showstringspaces=false,%
  literate={_}{\Underscore}1 {\%\%}{}0,%
  escapeinside={\#(}{\#)},%
  alsoletter={\#,\&},%
  keywords=[1]{not,from,import,def,if,else,elif,return,while,break,and,or,for,in,del,and,class,with,as,is,yield,async},%
  keywords=[2]{\#const,\#show,\#minimize,\#base,\#theory,\#count,\#external,\#program,\#script,\#end,\#heuristic,\#edge,\#project,\#show,\#sum},%
  keywords=[3]{&,&dom,&sum,&diff,&show},%
  morecomment=[l]{\#\ },%
  morecomment=[l]{\%\ },%
  morestring=[b]",%
  stringstyle={\itshape},%
  commentstyle={\color{darkgray}}%
}

\lstdefinelanguage{python}{%
  basicstyle=\ttfamily,%
  keywordstyle=[1]\bfseries,%
  showstringspaces=false,%
  literate={_}{\Underscore}{1},%
  escapeinside={\#(}{\#)},%
  alsoletter={\#,\&},%
  keywords=[1]{not,from,import,def,if,else,elif,return,while,break,and,or,for,in,del,and,class,with,as,is,yield,async},%
  morecomment=[l]{\#\ },%
  morestring=[b]",%
  stringstyle={\itshape},%
  commentstyle={\color{darkgray}}%
}


%     \renewcommand{\Underscore}{\textscale{1}{\textunderscore}}
%
% Without extra packages I had to provide an alternative underscore definition
% to get decent looking underscores.
%
% As an alternative to courier, it is also possible to use the `lmodern`
% package.
%
%     \usepackage[T1]{fontenc}
%     \usepackage{lmodern}
%     \usepackage{xcolor}
%     \usepackage{relsize}
%
%     % The language definitions below require the following packages:
%
%     \usepackage{courier}
%     \usepackage{xcolor}
%     \usepackage{relsize}
%
%     \input{listings}
%     \renewcommand{\Underscore}{\textscale{1}{\textunderscore}}
%
% Without extra packages I had to provide an alternative underscore definition
% to get decent looking underscores.
%
% As an alternative to courier, it is also possible to use the `lmodern`
% package.
%
%     \usepackage[T1]{fontenc}
%     \usepackage{lmodern}
%     \usepackage{xcolor}
%     \usepackage{relsize}
%
%     \input{listings}
%
% When the python or clingo language is given, tex code can be embedded
% escaping it in `#( ... #)`, which is treated like a comment in Python.
% Furthermore, for logic programs the character sequence is `%%` is not output.
% This can be used to add labels at the end of lines. For example, in a logic
% program one can write
%
%     H :- B.%%#( \label{lst:rule} #)
%
% to refer to the rule from the document without having to worry about changing
% line numbers when refactoring. The annotated program will still be runnable
% with clingo.

\providecommand{\Underscore}{\textunderscore}

\lstdefinelanguage{clingo}{%
  basicstyle=\ttfamily,%
  keywordstyle=[1]\bfseries,%
  keywordstyle=[2]\bfseries,%
  keywordstyle=[3]\bfseries,%
  showstringspaces=false,%
  literate={_}{\Underscore}1 {\%\%}{}0,%
  escapeinside={\#(}{\#)},%
  alsoletter={\#,\&},%
  keywords=[1]{not,from,import,def,if,else,elif,return,while,break,and,or,for,in,del,and,class,with,as,is,yield,async},%
  keywords=[2]{\#const,\#show,\#minimize,\#base,\#theory,\#count,\#external,\#program,\#script,\#end,\#heuristic,\#edge,\#project,\#show,\#sum},%
  keywords=[3]{&,&dom,&sum,&diff,&show},%
  morecomment=[l]{\#\ },%
  morecomment=[l]{\%\ },%
  morestring=[b]",%
  stringstyle={\itshape},%
  commentstyle={\color{darkgray}}%
}

\lstdefinelanguage{python}{%
  basicstyle=\ttfamily,%
  keywordstyle=[1]\bfseries,%
  showstringspaces=false,%
  literate={_}{\Underscore}{1},%
  escapeinside={\#(}{\#)},%
  alsoletter={\#,\&},%
  keywords=[1]{not,from,import,def,if,else,elif,return,while,break,and,or,for,in,del,and,class,with,as,is,yield,async},%
  morecomment=[l]{\#\ },%
  morestring=[b]",%
  stringstyle={\itshape},%
  commentstyle={\color{darkgray}}%
}


%
% When the python or clingo language is given, tex code can be embedded
% escaping it in `#( ... #)`, which is treated like a comment in Python.
% Furthermore, for logic programs the character sequence is `%%` is not output.
% This can be used to add labels at the end of lines. For example, in a logic
% program one can write
%
%     H :- B.%%#( \label{lst:rule} #)
%
% to refer to the rule from the document without having to worry about changing
% line numbers when refactoring. The annotated program will still be runnable
% with clingo.

\providecommand{\Underscore}{\textunderscore}

\lstdefinelanguage{clingo}{%
  basicstyle=\ttfamily,%
  keywordstyle=[1]\bfseries,%
  keywordstyle=[2]\bfseries,%
  keywordstyle=[3]\bfseries,%
  showstringspaces=false,%
  literate={_}{\Underscore}1 {\%\%}{}0,%
  escapeinside={\#(}{\#)},%
  alsoletter={\#,\&},%
  keywords=[1]{not,from,import,def,if,else,elif,return,while,break,and,or,for,in,del,and,class,with,as,is,yield,async},%
  keywords=[2]{\#const,\#show,\#minimize,\#base,\#theory,\#count,\#external,\#program,\#script,\#end,\#heuristic,\#edge,\#project,\#show,\#sum},%
  keywords=[3]{&,&dom,&sum,&diff,&show},%
  morecomment=[l]{\#\ },%
  morecomment=[l]{\%\ },%
  morestring=[b]",%
  stringstyle={\itshape},%
  commentstyle={\color{darkgray}}%
}

\lstdefinelanguage{python}{%
  basicstyle=\ttfamily,%
  keywordstyle=[1]\bfseries,%
  showstringspaces=false,%
  literate={_}{\Underscore}{1},%
  escapeinside={\#(}{\#)},%
  alsoletter={\#,\&},%
  keywords=[1]{not,from,import,def,if,else,elif,return,while,break,and,or,for,in,del,and,class,with,as,is,yield,async},%
  morecomment=[l]{\#\ },%
  morestring=[b]",%
  stringstyle={\itshape},%
  commentstyle={\color{darkgray}}%
}


%     \renewcommand{\Underscore}{\textscale{1}{\textunderscore}}
%
% Without extra packages I had to provide an alternative underscore definition
% to get decent looking underscores.
%
% As an alternative to courier, it is also possible to use the `lmodern`
% package.
%
%     \usepackage[T1]{fontenc}
%     \usepackage{lmodern}
%     \usepackage{xcolor}
%     \usepackage{relsize}
%
%     % The language definitions below require the following packages:
%
%     \usepackage{courier}
%     \usepackage{xcolor}
%     \usepackage{relsize}
%
%     % The language definitions below require the following packages:
%
%     \usepackage{courier}
%     \usepackage{xcolor}
%     \usepackage{relsize}
%
%     \input{listings}
%     \renewcommand{\Underscore}{\textscale{1}{\textunderscore}}
%
% Without extra packages I had to provide an alternative underscore definition
% to get decent looking underscores.
%
% As an alternative to courier, it is also possible to use the `lmodern`
% package.
%
%     \usepackage[T1]{fontenc}
%     \usepackage{lmodern}
%     \usepackage{xcolor}
%     \usepackage{relsize}
%
%     \input{listings}
%
% When the python or clingo language is given, tex code can be embedded
% escaping it in `#( ... #)`, which is treated like a comment in Python.
% Furthermore, for logic programs the character sequence is `%%` is not output.
% This can be used to add labels at the end of lines. For example, in a logic
% program one can write
%
%     H :- B.%%#( \label{lst:rule} #)
%
% to refer to the rule from the document without having to worry about changing
% line numbers when refactoring. The annotated program will still be runnable
% with clingo.

\providecommand{\Underscore}{\textunderscore}

\lstdefinelanguage{clingo}{%
  basicstyle=\ttfamily,%
  keywordstyle=[1]\bfseries,%
  keywordstyle=[2]\bfseries,%
  keywordstyle=[3]\bfseries,%
  showstringspaces=false,%
  literate={_}{\Underscore}1 {\%\%}{}0,%
  escapeinside={\#(}{\#)},%
  alsoletter={\#,\&},%
  keywords=[1]{not,from,import,def,if,else,elif,return,while,break,and,or,for,in,del,and,class,with,as,is,yield,async},%
  keywords=[2]{\#const,\#show,\#minimize,\#base,\#theory,\#count,\#external,\#program,\#script,\#end,\#heuristic,\#edge,\#project,\#show,\#sum},%
  keywords=[3]{&,&dom,&sum,&diff,&show},%
  morecomment=[l]{\#\ },%
  morecomment=[l]{\%\ },%
  morestring=[b]",%
  stringstyle={\itshape},%
  commentstyle={\color{darkgray}}%
}

\lstdefinelanguage{python}{%
  basicstyle=\ttfamily,%
  keywordstyle=[1]\bfseries,%
  showstringspaces=false,%
  literate={_}{\Underscore}{1},%
  escapeinside={\#(}{\#)},%
  alsoletter={\#,\&},%
  keywords=[1]{not,from,import,def,if,else,elif,return,while,break,and,or,for,in,del,and,class,with,as,is,yield,async},%
  morecomment=[l]{\#\ },%
  morestring=[b]",%
  stringstyle={\itshape},%
  commentstyle={\color{darkgray}}%
}


%     \renewcommand{\Underscore}{\textscale{1}{\textunderscore}}
%
% Without extra packages I had to provide an alternative underscore definition
% to get decent looking underscores.
%
% As an alternative to courier, it is also possible to use the `lmodern`
% package.
%
%     \usepackage[T1]{fontenc}
%     \usepackage{lmodern}
%     \usepackage{xcolor}
%     \usepackage{relsize}
%
%     % The language definitions below require the following packages:
%
%     \usepackage{courier}
%     \usepackage{xcolor}
%     \usepackage{relsize}
%
%     \input{listings}
%     \renewcommand{\Underscore}{\textscale{1}{\textunderscore}}
%
% Without extra packages I had to provide an alternative underscore definition
% to get decent looking underscores.
%
% As an alternative to courier, it is also possible to use the `lmodern`
% package.
%
%     \usepackage[T1]{fontenc}
%     \usepackage{lmodern}
%     \usepackage{xcolor}
%     \usepackage{relsize}
%
%     \input{listings}
%
% When the python or clingo language is given, tex code can be embedded
% escaping it in `#( ... #)`, which is treated like a comment in Python.
% Furthermore, for logic programs the character sequence is `%%` is not output.
% This can be used to add labels at the end of lines. For example, in a logic
% program one can write
%
%     H :- B.%%#( \label{lst:rule} #)
%
% to refer to the rule from the document without having to worry about changing
% line numbers when refactoring. The annotated program will still be runnable
% with clingo.

\providecommand{\Underscore}{\textunderscore}

\lstdefinelanguage{clingo}{%
  basicstyle=\ttfamily,%
  keywordstyle=[1]\bfseries,%
  keywordstyle=[2]\bfseries,%
  keywordstyle=[3]\bfseries,%
  showstringspaces=false,%
  literate={_}{\Underscore}1 {\%\%}{}0,%
  escapeinside={\#(}{\#)},%
  alsoletter={\#,\&},%
  keywords=[1]{not,from,import,def,if,else,elif,return,while,break,and,or,for,in,del,and,class,with,as,is,yield,async},%
  keywords=[2]{\#const,\#show,\#minimize,\#base,\#theory,\#count,\#external,\#program,\#script,\#end,\#heuristic,\#edge,\#project,\#show,\#sum},%
  keywords=[3]{&,&dom,&sum,&diff,&show},%
  morecomment=[l]{\#\ },%
  morecomment=[l]{\%\ },%
  morestring=[b]",%
  stringstyle={\itshape},%
  commentstyle={\color{darkgray}}%
}

\lstdefinelanguage{python}{%
  basicstyle=\ttfamily,%
  keywordstyle=[1]\bfseries,%
  showstringspaces=false,%
  literate={_}{\Underscore}{1},%
  escapeinside={\#(}{\#)},%
  alsoletter={\#,\&},%
  keywords=[1]{not,from,import,def,if,else,elif,return,while,break,and,or,for,in,del,and,class,with,as,is,yield,async},%
  morecomment=[l]{\#\ },%
  morestring=[b]",%
  stringstyle={\itshape},%
  commentstyle={\color{darkgray}}%
}


%
% When the python or clingo language is given, tex code can be embedded
% escaping it in `#( ... #)`, which is treated like a comment in Python.
% Furthermore, for logic programs the character sequence is `%%` is not output.
% This can be used to add labels at the end of lines. For example, in a logic
% program one can write
%
%     H :- B.%%#( \label{lst:rule} #)
%
% to refer to the rule from the document without having to worry about changing
% line numbers when refactoring. The annotated program will still be runnable
% with clingo.

\providecommand{\Underscore}{\textunderscore}

\lstdefinelanguage{clingo}{%
  basicstyle=\ttfamily,%
  keywordstyle=[1]\bfseries,%
  keywordstyle=[2]\bfseries,%
  keywordstyle=[3]\bfseries,%
  showstringspaces=false,%
  literate={_}{\Underscore}1 {\%\%}{}0,%
  escapeinside={\#(}{\#)},%
  alsoletter={\#,\&},%
  keywords=[1]{not,from,import,def,if,else,elif,return,while,break,and,or,for,in,del,and,class,with,as,is,yield,async},%
  keywords=[2]{\#const,\#show,\#minimize,\#base,\#theory,\#count,\#external,\#program,\#script,\#end,\#heuristic,\#edge,\#project,\#show,\#sum},%
  keywords=[3]{&,&dom,&sum,&diff,&show},%
  morecomment=[l]{\#\ },%
  morecomment=[l]{\%\ },%
  morestring=[b]",%
  stringstyle={\itshape},%
  commentstyle={\color{darkgray}}%
}

\lstdefinelanguage{python}{%
  basicstyle=\ttfamily,%
  keywordstyle=[1]\bfseries,%
  showstringspaces=false,%
  literate={_}{\Underscore}{1},%
  escapeinside={\#(}{\#)},%
  alsoletter={\#,\&},%
  keywords=[1]{not,from,import,def,if,else,elif,return,while,break,and,or,for,in,del,and,class,with,as,is,yield,async},%
  morecomment=[l]{\#\ },%
  morestring=[b]",%
  stringstyle={\itshape},%
  commentstyle={\color{darkgray}}%
}


%
% When the python or clingo language is given, tex code can be embedded
% escaping it in `#( ... #)`, which is treated like a comment in Python.
% Furthermore, for logic programs the character sequence is `%%` is not output.
% This can be used to add labels at the end of lines. For example, in a logic
% program one can write
%
%     H :- B.%%#( \label{lst:rule} #)
%
% to refer to the rule from the document without having to worry about changing
% line numbers when refactoring. The annotated program will still be runnable
% with clingo.

\providecommand{\Underscore}{\textunderscore}

\lstdefinelanguage{clingo}{%
  basicstyle=\ttfamily,%
  keywordstyle=[1]\bfseries,%
  keywordstyle=[2]\bfseries,%
  keywordstyle=[3]\bfseries,%
  showstringspaces=false,%
  literate={_}{\Underscore}1 {\%\%}{}0,%
  escapeinside={\#(}{\#)},%
  alsoletter={\#,\&},%
  keywords=[1]{not,from,import,def,if,else,elif,return,while,break,and,or,for,in,del,and,class,with,as,is,yield,async},%
  keywords=[2]{\#const,\#show,\#minimize,\#base,\#theory,\#count,\#external,\#program,\#script,\#end,\#heuristic,\#edge,\#project,\#show,\#sum},%
  keywords=[3]{&,&dom,&sum,&diff,&show},%
  morecomment=[l]{\#\ },%
  morecomment=[l]{\%\ },%
  morestring=[b]",%
  stringstyle={\itshape},%
  commentstyle={\color{darkgray}}%
}

\lstdefinelanguage{python}{%
  basicstyle=\ttfamily,%
  keywordstyle=[1]\bfseries,%
  showstringspaces=false,%
  literate={_}{\Underscore}{1},%
  escapeinside={\#(}{\#)},%
  alsoletter={\#,\&},%
  keywords=[1]{not,from,import,def,if,else,elif,return,while,break,and,or,for,in,del,and,class,with,as,is,yield,async},%
  morecomment=[l]{\#\ },%
  morestring=[b]",%
  stringstyle={\itshape},%
  commentstyle={\color{darkgray}}%
}



\begin{frame}
\titlepage
\end{frame}

\begin{frame}{Recap}
\begin{itemize}
\item \emph{Logic of Here-and-There with Constraints (\HTC)} allows for capturing non-monotonic reasoning over arbitrary constraints involving \emph{constraint variables} and \emph{conditional expressions}
\pause
\item Flexible syntax, \emph{constraint atoms} are (infinite) strings of symbols that may refer to variables and domain elements
\begin{align*}
"x-y\leq d"\quad"a"\quad"\mathit{sum}\agg{ \, s_1, s_2, \dots \,  } = s_0"
\end{align*}
\vspace*{-.5cm}
\pause
\item Flexible semantics, \emph{denotations} assign constraint atoms sets of satisfying \emph{valuations}
\begin{align*}
  \den{x-y\leq d}
  =
  \{v\in\mathcal{V}\mid v(x),v(y), d\in \mathbb{Z}, v(x)-v(y)\leq d\}
\end{align*}
\end{itemize}
\end{frame}

\begin{frame}{Recap}
\begin{itemize}
\item \emph{Conditional constraint atoms} may contain \emph{conditional expressions} $(\ctermm{s}{s'}{\varphi})$
\pause
\item \emph{Evaluation function} replaces conditional expressions $\tau=(\ctermm{s}{s'}{\varphi})$:
  \begin{align*}
    \eval{h}{t}(\tau)
    &=
      \left\{
      \begin{array}{ll}
        s & \text{if } \langle h,t\rangle\models\varphi\\
        s' & \text{if } \langle t,t\rangle \hspace{2pt}\not\models\varphi\\
        \undefined & \text{otherwise}
      \end{array}
      \right.
  \end{align*}
\vspace*{-.5cm}
\pause
\item \emph{Interpretations} $\tuple{h,t}$ are pairs of valuations with $h\subseteq t$
\pause
\item Syntax, semantics of \emph{formulas} are standard \HT\ but for condition:
$\langle h,t \rangle \models c \text{ if } h\in \den{\eval{h}{t}(c)}$
\end{itemize}
\end{frame}

\begin{frame}{Motivation}
\begin{align*}
  \mathit{total}(R) := \mathit{sum}\agg{ \, \mathit{tax}(P) : \mathit{lives}(P,R) \,  } \ \leftarrow \ \mathit{region}(R)
\end{align*}
\pause
\small\lstinputlisting[linerange={21-21},numbers=none,mathescape=true,basicstyle={\ttfamily},language=clingo]{examples/taxes.lp}
\quad
\pause
\begin{itemize}
  \item Linear, modular translation to CASP using the \clingo\ Python API
  \pause
  \item Python version of \clingcon\ as solver
  \pause
  \item Bonus goal: implementation only by adding atoms and rules for each theory atom
\end{itemize}
\end{frame}

\begin{frame}
\tableofcontents
\end{frame}

\section{\HTC\ to \emph{CASP}: Syntactic and semantic differences}

\begin{frame}{Input language}
\scriptsize\lstinputlisting[mathescape=true,basicstyle={\ttfamily},language=clingo]{examples/htc_lang.lp}
\end{frame}

\begin{frame}{Example}
\scriptsize\lstinputlisting[linerange={13-29},mathescape=true,basicstyle={\ttfamily},language=clingo]{examples/taxes.lp}
\end{frame}

\begin{frame}{Target language}
\scriptsize\lstinputlisting[mathescape=true,escapeinside={\#(}{\#)},basicstyle={\ttfamily},language=clingo]{examples/cp_lang.lp}
\end{frame}

\begin{frame}{Semantic differences}
  \begin{itemize}
    \item \emph{CASP} does not allow variables to be undefined 
    \pause
    \item \clingcon\ treats theory atoms in the body as \emph{strict, external} and theory atoms in the head as \emph{non-strict, defined}
    \pause
    \item \HTC\ stable models require values of variables to be \emph{founded}, variables may be undefined, no implicit choice for body atoms
    \pause
    \item
    \lstinputlisting[linerange={1-1},mathescape=true,escapeinside={\#(}{\#)},basicstyle={\ttfamily},language=clingo]{examples/semantics.lp}
    \pause
    \begin{tabular}{ll}
    \clingcon : & $\{\{(\mathtt{x},v)\} \mid v \ne 1 \} \cup \{\{\mathtt{a}, (\mathtt{x},1)\}\}$\\
    \pause
    \HTC      : & $\{\emptyset\}$
    \end{tabular}
    \pause
    \item Add rules, sum constraints and auxiliary variables to implement constructs foreign to language of \clingcon
  \end{itemize}
\end{frame}


\begin{frame}{Preprocessing: Seperate head from body}
  \begin{itemize}
    \item Use the abstract syntax tree to mark body and head theory atoms while parsing
    \pause
    \[
\mathtt{\&}t\mathtt{\{}e_1:c_1,\dots,e_n:c_n\mathtt{\}} \prec k \Rightarrow \mathtt{\&}t\mathtt{(}\mathit{l}\mathtt{)}\mathtt{\{}e_1:c_1,\dots,e_n:c_n\mathtt{\}}\prec k
    \]
    where $t$ is the name of the theory atom, $e_i$ are tuples of symbols, $c_i$ are conditions (both non-ground), $k$ is a theory term, $\prec$ is a relation symbol and $l\in\{\mathtt{head},\mathtt{body}\}$ 
    \pause
    \item 
    \lstinputlisting[linerange={1-1},mathescape=true,escapeinside={\#(}{\#)},basicstyle={\ttfamily},language=clingo]{examples/semantics.lp}
    $\Rightarrow$
    \lstinputlisting[linerange={2-2},mathescape=true,escapeinside={\#(}{\#)},basicstyle={\ttfamily},language=clingo]{examples/semantics.lp}
  \end{itemize}
\end{frame}

\section{Founding variable values}

\begin{frame}{Define variables}
  \begin{itemize}
    \item For every variable \texttt{x} occurring in the program, capture definedness with atom \texttt{def(x)} and add the rule
\lstinputlisting[linerange={7-7},mathescape=true,escapeinside={\#(}{\#)},basicstyle={\ttfamily},language=clingo]{examples/semantics.lp}
    \pause
    \item For every condition-free linear constraint atom 
    \lstinputlisting[linerange={3-3},mathescape=true,escapeinside={\#(}{\#)},basicstyle={\ttfamily},language=clingo]{examples/semantics.lp}
       add rule
       \lstinputlisting[linerange={4-4},mathescape=true,escapeinside={\#(}{\#)},basicstyle={\ttfamily},language=clingo]{examples/semantics.lp}
       for every $\mathtt{x} \in \vars{e_1}\cup\dots\cup\vars{e_n}$
   \end{itemize}
\end{frame}

\begin{frame}{Translating linear constraint atoms in the body}
  For every condition-free linear constraint atom 
    \lstinputlisting[linerange={8-8},mathescape=true,escapeinside={\#(}{\#)},basicstyle={\ttfamily},language=clingo]{examples/semantics.lp} 
    \pause
    \begin{itemize}
      \item  Introduce linear constraint atom  
            \lstinputlisting[linerange={5-5},mathescape=true,escapeinside={\#(}{\#)},basicstyle={\ttfamily},language=clingo]{examples/semantics.lp}
            \pause
      \item Do not communicate original linear constraint to \clingcon\ \pause 
      \item Add rules
      \only<4>{\lstinputlisting[linerange={9-12},mathescape=true,escapeinside={\#(}{\#)},basicstyle={\ttfamily},language=clingo]{examples/semantics.lp}}
      \only<5>{\lstinputlisting[linerange={13-15},mathescape=true,escapeinside={\#(}{\#)},basicstyle={\ttfamily},language=clingo]{examples/semantics.lp}}
      where $\{\mathtt{x}_1,\dots,\mathtt{x}_n\}=\vars{e_1}\cup\dots\cup\vars{e_n}$
    \end{itemize}
\end{frame}

\begin{frame}{Example}
  We translate
    \lstinputlisting[linerange={1-1},mathescape=true,escapeinside={\#(}{\#)},basicstyle={\ttfamily},language=clingo]{examples/semantics.lp}
    \pause
      to
    \only<2>{\lstinputlisting[linerange={16-19},mathescape=true,escapeinside={\#(}{\#)},basicstyle={\ttfamily},language=clingo]{examples/semantics.lp}}
    \only<3->{\lstinputlisting[linerange={20-23},mathescape=true,escapeinside={\#(}{\#)},basicstyle={\ttfamily},language=clingo]{examples/semantics.lp}}
    \pause
    \pause
    \begin{tabular}{ll}
    \clingcon : & $\{\{(x,0)\}\}$\\
    \pause
    \HTC      : & $\{\emptyset\}$
    \end{tabular}
\end{frame}

\section{Translating conditional terms}
\begin{frame}{Translating conditional terms}
    \[
\mathtt{\&}t(l)\mathtt{\{}\dots,e_1:c_1,\dots,e_n:c_n,\dots\mathtt{\}} \prec k
    \]
    \vspace*{-.3cm}
    \pause
    \begin{itemize}
  \item For every $e_i:c_i$, introduce auxiliary variable $\mathit{aux}_i$ and add following rules where $\{\mathtt{x}_1,\dots,\mathtt{x}_n\}=\vars{e_i}$
  \lstinputlisting[linerange={1-5},mathescape=true,escapeinside={\#(}{\#)},basicstyle={\ttfamily},language=clingo]{examples/conditional.lp}
  
  \pause
  \item Replace conditional terms with respective auxiliary variables
  \small\lstinputlisting[linerange={6-9},mathescape=true,escapeinside={\#(}{\#)},basicstyle={\ttfamily},language=clingo]{examples/conditional.lp}
    \end{itemize}
\end{frame}

\begin{frame}{Example: Vicious cycle}
  \only<1>{\lstinputlisting[linerange={1-2},mathescape=true,escapeinside={\#(}{\#)},basicstyle={\ttfamily},language=clingo]{examples/vicious.lp}}
  \only<2>{\scriptsize\lstinputlisting[linerange={3-21},mathescape=true,escapeinside={\#(}{\#)},basicstyle={\ttfamily},language=clingo]{examples/vicious.lp}}
\end{frame}

\section{Translating assignments, range, minimum/maximum constraint}
\begin{frame}{Translating assignments and range constraints}
 \begin{itemize}
    \item For assignment $\mathtt{\&}t\mathtt{\{}e_1,\dots,e_n\mathtt{\}} =: x$, we add
    \pause
    \lstinputlisting[linerange={1-2},mathescape=true,escapeinside={\#(}{\#)},basicstyle={\ttfamily},language=clingo]{examples/assignments.lp}
    where $\{\mathtt{x}_1,\dots,\mathtt{x}_n\}=\vars{e_1}\cup\dots\cup\vars{e_n}$
    \pause
    \item For range constraint $\mathtt{\&in}\mathtt{\{}l\mathtt{..}u{\}} = x$, we add
    \pause
    \lstinputlisting[linerange={3-4},mathescape=true,escapeinside={\#(}{\#)},basicstyle={\ttfamily},language=clingo]{examples/assignments.lp}
 \end{itemize}
\end{frame}

\begin{frame}{Translating minimum constraints}
\[\mathtt{\&min}(l)\mathtt{\{}e_1,\dots,e_n\mathtt{\}} \prec k\]
\pause
 \begin{itemize}
    \item Add auxiliary variable $\mathit{min}$ and for every $e_i$, add following rules where $\{\mathtt{x}_1,\dots,\mathtt{x}_n\}=\vars{e_i}$ 
    \lstinputlisting[linerange={1-3},mathescape=true,escapeinside={\#(}{\#)},basicstyle={\ttfamily},language=clingo]{examples/minmax.lp}
    \item Check whether a correct value was selected and create linear constraint atom
    \lstinputlisting[linerange={4-},mathescape=true,escapeinside={\#(}{\#)},basicstyle={\ttfamily},language=clingo]{examples/minmax.lp}
 \end{itemize}
\end{frame}

\section{Demonstration}

\end{document}